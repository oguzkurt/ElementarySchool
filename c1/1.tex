\documentclass{ximera}
%% You can put user macros here
%% However, you cannot make new environments

\usepackage[letterpaper, total={6in, 8in}]{geometry}
\graphicspath{{./}{firstExample/}{secondExample/}}

\usepackage{tikz}
\usepackage{tkz-euclide}
\usetkzobj{all}

\tikzstyle geometryDiagrams=[ultra thick,color=blue!50!black]

%\usepackage{enumerate}
\usepackage{euler}
\usepackage{manfnt}
\usepackage{MnSymbol}%

\usepackage{geometry}
 \geometry{
 letterpaper,
 left=18mm,
 right=18mm,
 top=20mm,
 bottom=14mm
 }
 
 
\usetikzlibrary{shapes,snakes}
\tikzset{
  dot hidden/.style={},
  line hidden/.style={},
  dot colour/.style={dot hidden/.append style={color=#1}},
  dot colour/.default=black,
  line colour/.style={line hidden/.append style={color=#1}},
  line colour/.default=black
}

%%%%%%%%%%%%%%% DICE! %%%%%%%%%%%%%%%%%
\def\dotsize{0.1}

\usepackage{xparse}
\tikzset{
  dot hidden/.style={},
  line hidden/.style={},
  dot colour/.style={dot hidden/.append style={color=#1}},
  dot colour/.default=black,
  line colour/.style={line hidden/.append style={color=#1}},
  line colour/.default=black
}

\usepackage{xparse}
\NewDocumentCommand{\drawdie}{O{}m}{
\begin{tikzpicture}[x=2em,y=2em,radius=0.1,#1]
  \draw[rounded corners=0.5,line hidden] (0,0) rectangle (1,1);
  \ifodd#2
    \fill[dot hidden] (0.5,0.5) circle;
  \fi
  \ifnum#2>1
    \fill[dot hidden] (0.2,0.2) circle;
    \fill[dot hidden] (0.8,0.8) circle;
   \ifnum#2>3
     \fill[dot hidden] (0.2,0.8) circle;
     \fill[dot hidden] (0.8,0.2) circle;
    \ifnum#2>5
      \fill[dot hidden] (0.8,0.5) circle;
      \fill[dot hidden] (0.2,0.5) circle;
     \ifnum#2>7
       \fill[dot hidden] (0.5,0.8) circle;
       \fill[dot hidden] (0.5,0.2) circle;
      \fi
    \fi
  \fi
\fi
\end{tikzpicture}
}   

%%%%%%%%%%%%%%%%%

%%%%%%%%%%%%%

\newcommand{\Hsquare}{
\boxed{\begin{matrix} ~~~~~~~ \Huge\, \end{matrix}}
}

%%%%%%%%%%%%%%%%%%%%%% MENTAL ADD BEGIN%%%
\tikzset{
mycircle/.pic={
\draw[dashed] (0,0) circle (1.6cm); 
}
}

\tikzset{
myellipse/.pic={
\draw[dashed] (0,0) ellipse (3.2cm and 1.6cm); 
}
}

\tikzset{
mybox/.pic={
\draw[line width=0.35mm] (-1.6,-.8) rectangle (1.6,.8);  
}  
}

\newcommand{\mentaladd}[2]{
\begin{tikzpicture}
  \matrix[column sep=3mm, row sep=3mm,ampersand replacement=\&]{
      \node[]{\Large$\mathbf{#1}$}; \& \node[]{\Large $\mathbf{+}$}; \& \node[]{\Large $\mathbf{#2}$}; \&\node[]{\Large $\mathbf{=}$}; \& \pic{mybox} ; \\
      \pic{mycircle}; \& \node[]{\Large +};\& \pic{mycircle}; \&\node[]{\Large=}; \& \pic{myellipse}; \\
    };
\end{tikzpicture}
}
%%%%%%%%%%%%%%%%%%MENTAL ADD END %%%%%%%%%%%%%

%%%%%%%%%%%%%%%%%%%% KINDERGARTEN WRITING LINE 

\usepackage{scalerel}
\usepackage[usestackEOL]{stackengine}
\newlength\letterheight
\setlength\letterheight{.5in}
\def\midpitch{.6}
\parskip 0.3in
\def\stacktype{L}\def\stackalignment{l}
%From morsburg at http://tex.stackexchange.com/questions/12537/
%how-can-i-make-a-horizontal-dashed-line/12553#12553
\def\dashfill{\cleaders\hbox to .6em{-}\hfill}
\newcommand\dashline[1]{\abovebaseline[-2pt]{\hbox to #1{\dashfill\hfil}}}
\def\myhline{\rule{\textwidth}{.3pt}}
\newcommand\blankrow{%
  \setstackgap{L}{\the\letterheight}%
  \stackon[\midpitch\letterheight]{%
    \stackon{\myhline}{\myhline}}{\dashline{\textwidth}}%
}
\newcommand\fillrow[1]{%
  \stackinset{l}{}{b}{}%
  {\scalerel*{\rule{0pt}{.85\ht\strutbox}\smash{#1}}{\blankrow}}{\blankrow}}
%%%%%%%%%%%%%%%%%%%%%%%% END KINDERGARTEN

\title{Earth -- Expansion HW \hfill \makebox[0.5\textwidth]{Name:\enspace\hrulefill}}
\author{Oguz Kurt}
\begin{document}
\begin{abstract}
\empty
\end{abstract}
%maketitle
%%%% ADDITION TABLES %%%%%%%%% USE: \sagestr{additiontable(m,x)}
\begin{sagesilent}
set_random_seed(1)

def additiontable(m,x):
    # start of the table
    s  = [r" $$\begin{array}{|r|r|r|r|r|} "]
    s.append(r" \hline ")
    for i in range(4):
        s.append(r"  &+{0}".format(m*10**i))
    s.append(r" \\ \hline ")
    for k in x:
        s.append(r" {0}".format(k))
        for i in range(4):
            s.append(r" & \answer{ ")
            s.append(r" {0}".format(k+m*10**(i)))
            s.append(r" } ")
        s.append(r" \\ \hline ")   
    s.append(r" \end{array}$$ ")
    return ''.join(s)


def in_out(m,x,side):
    if side=='r':
        s  = [r" $$\begin{array}{|r|r|}"]
        s.append(r" \hline ")
        s.append(r" \text{IN} & \text{OUT} \\ \hline ")
        for a in x:
            s.append(r" {0}".format(a))
            s.append(r" & \answer{ ")
            s.append(r" {0}".format(a+m))
            s.append(r" } \\ \hline  ")
        s.append(r" \text{RULE:} & ") 
        if m<0:
            s.append(r" {0}".format(m))
        else:
            s.append(r" +"+r" {0}".format(m))
        s.append(r" \\ \hline \end{array}$$ ")
    if side=='rb':
        s  = [r" $$\begin{array}{|r|r|}"]
        s.append(r" \hline ")
        s.append(r" \text{IN} & \text{OUT} \\ \hline ")
        for myindex in range(len(x)):
            a=x[myindex]
            if myindex==0:
                s.append(r" {0}".format(a))
                s.append(r" & ")
                s.append(r" {0}".format(a+m))
                s.append(r"  \\ \hline  ")
            else:
                s.append(r" {0}".format(a))
                s.append(r" & \answer{ ")
                s.append(r" {0}".format(a+m))
                s.append(r" } \\ \hline  ")
        s.append(r" \text{RULE:} & ")
        s.append(r" +\answer{ ")
        s.append(r" {0}".format(m))
        s.append(r" } \\ \hline \end{array}$$ ")
    if side=='l':    
        s  = [r" $$\begin{array}{|r|r|}"]
        s.append(r" \hline ")
        s.append(r" \text{IN} & \text{OUT} \\ \hline ")
        for a in x:
            s.append(r" \answer{ ")
            s.append(r" {0}".format(a))
            s.append(r" } & ")
            s.append(r" {0}".format(a+m))
            s.append(r" \\ \hline  ")
        s.append(r" \text{RULE:} & ") 
        if m<0:
            s.append(r" {0}".format(m))
        else:
            s.append(r" +"+r" {0}".format(m))
        s.append(r" \\ \hline \end{array}$$ ")
    if side=="lb":
        s  = [r" $$\begin{array}{|r|r|}"]
        s.append(r" \hline ")
        s.append(r" \text{IN} & \text{OUT} \\ \hline ")
        for myindex in range(len(x)):
            a=x[myindex]
            if myindex==0:
                s.append(r" {0}".format(a))
                s.append(r" & ")
                s.append(r" {0}".format(a+m))
                s.append(r"  \\ \hline  ")
            else:
                s.append(r" \answer{ ")
                s.append(r" {0}".format(a))
                s.append(r" } & ")
                s.append(r" {0}".format(a+m))
                s.append(r" \\ \hline  ")
        s.append(r" \text{RULE:} & ")
        s.append(r" +\answer{ ")
        s.append(r" {0}".format(m))
        s.append(r" } \\ \hline \end{array}$$ ")
    if side=="m":
        s  = [r" $$\begin{array}{|r|r|}"]
        s.append(r" \hline ")
        s.append(r" \text{IN} & \text{OUT} \\ \hline ")
        for a in x:
            coin=randint(1,2)
            if coin==1:
                s.append(r" \answer{ ")
                s.append(r" {0}".format(a))
                s.append(r" } & ")
                s.append(r" {0}".format(a+m))
                s.append(r" \\ \hline  ")
            if coin==2:
                s.append(r" {0}".format(a))
                s.append(r" & \answer{ ")
                s.append(r" {0}".format(a+m))
                s.append(r" } \\ \hline  ")
        s.append(r" \text{RULE:} & ") 
        if m<0:
            s.append(r" {0}".format(m))
        else:
            s.append(r" +"+r" {0}".format(m))
        s.append(r" \\ \hline \end{array}$$ ")
        
    return ''.join(s)
\end{sagesilent}

\begin{problem}
\begin{sagesilent}
m=randint(3,9)
i=randint(0,3)
n=m*10**i
k=randint(4,7)
x=[randint(5,10000) for i in range(k)]
\end{sagesilent}
Please, complete the following addition table:
\sagestr{in_out(n,x,"r" )}
\end{problem}

\begin{problem}
\begin{sagesilent}
m=randint(3,9)
i=randint(0,3)
n=m*10**i
k=randint(4,7)
x=[randint(5,10000) for i in range(k)]
\end{sagesilent}
Please, complete the following addition table:
\sagestr{in_out(n,x,"r")}
\end{problem}

\begin{problem}
\begin{sagesilent}
m=randint(3,9)
i=randint(0,3)
n=m*10**i
k=randint(4,7)
x=[randint(5,10000) for i in range(k)]
\end{sagesilent}
Please, complete the following addition table:
\sagestr{in_out(n,x,"rb")}
\end{problem}

\begin{problem}
\begin{sagesilent}
m=randint(3,9)
i=randint(0,3)
n=m*10**i
k=randint(4,7)
x=[randint(5,10000) for i in range(k)]
\end{sagesilent}
Please, complete the following addition table:
\sagestr{in_out(n,x,"rb")}
\end{problem}


\begin{problem}
\begin{sagesilent}
m=randint(3,9)
i=randint(0,3)
n=m*10**i
k=randint(4,7)
x=[randint(5,10000) for i in range(k)]
\end{sagesilent}
Please, complete the following addition table:
\sagestr{in_out(n,x,"l")}
\end{problem}


\begin{problem}
\begin{sagesilent}
m=randint(3,9)
i=randint(0,3)
n=m*10**i
k=randint(4,7)
x=[randint(5,10000) for i in range(k)]
\end{sagesilent}
Please, complete the following addition table:
\sagestr{in_out(n,x,"l")}
\end{problem}
\begin{problem}
\begin{sagesilent}
m=randint(3,9)
i=randint(0,3)
n=m*10**i
k=randint(4,7)
x=[randint(5,10000) for i in range(k)]
\end{sagesilent}
Please, complete the following addition table:
\sagestr{in_out(n,x,"lb")}
\end{problem}

\begin{problem}
\begin{sagesilent}
m=randint(3,9)
i=randint(0,3)
n=m*10**i
k=randint(4,7)
x=[randint(5,10000) for i in range(k)]
\end{sagesilent}
Please, complete the following addition table:
\sagestr{in_out(n,x,"lb")}
\end{problem}

\begin{problem}
\begin{sagesilent}
m=randint(3,9)
i=randint(0,3)
n=m*10**i
k=randint(4,7)
x=[randint(5,10000) for i in range(k)]
\end{sagesilent}
Please, complete the following addition table:
\sagestr{in_out(n,x,"m")}
\end{problem}

\begin{problem}
\begin{sagesilent}
m=randint(3,9)
i=randint(0,3)
n=m*10**i
k=randint(4,7)
x=[randint(5,10000) for i in range(k)]
\end{sagesilent}
Please, complete the following addition table:
\sagestr{in_out(n,x,"m")}
\end{problem}


\begin{problem}
\begin{sagesilent}
m=randint(3,9)
k=randint(3,7)
x=[randint(5,9999) for i in range(k)]
\end{sagesilent}
Please, complete the following addition table:
\sagestr{additiontable(m,x)}
\end{problem}


\begin{problem}
\begin{sagesilent}
m=randint(3,9)
k=randint(3,7)
x=[randint(5,9999) for i in range(k)]
\end{sagesilent}
Please, complete the following addition table:
\sagestr{additiontable(m,x)}
\end{problem}


\begin{problem}
\begin{sagesilent}
m=randint(3,9)
k=randint(3,7)
x=[randint(5,9999) for i in range(k)]
\end{sagesilent}
Please, complete the following addition table:
\sagestr{additiontable(m,x)}
\end{problem}

\begin{problem}
\begin{sagesilent}
m=randint(3,9)
k=randint(3,7)
x=[randint(5,9999) for i in range(k)]
\end{sagesilent}
Please, complete the following addition table:
\sagestr{additiontable(m,x)}
\end{problem}





\end{document}
\begin{problem}
Please, fill in the blanks on the following addition table.
\begin{center}
$$
%\arraycolsep=5pt\def\arraystretch{1.8}
\begin{array}[s]{|r||r|r|r|r|}
\hline
&~~~~~~~ \mathbf{+5} &~~~~~ \mathbf{+50} &~~~~ \mathbf{+500} &~~~ \mathbf{+5000} 
\\ 
\hline
\hline
\mathbf{67} &\answer{\sage{67+5}}&\answer{\sage{67+50}}&\answer{\sage{67+500}}&\answer{\sage{67+5000}} 
\\
\hline
\mathbf{119} &\answer{\sage{119+5}}&\answer{\sage{119+50}}&\answer{\sage{119+500}}&\answer{\sage{119+5000}} 
\\
\hline
\mathbf{663} &\answer{\sage{663+5}}&\answer{\sage{663+50}}&\answer{\sage{663+500}}&\answer{\sage{663+5000}} 
\\
\hline
\end{array}
$$
\end{center}
\end{problem}

\begin{problem}
Please, fill in the blanks on the following addition table.

\end{problem}
\end{document}
\begin{problem}
Please, complete the addition table. Note that the students are expected to think in reverse order.

%\setlength\extrarowheight{12pt}
$
\begin{array}{|r|r|}
\hline
 IN & OUT \\
 \hline
 5 & 7 \\
 \hline
\answer{\sage{11-2}} & 11 \\
 \hline
 \answer{\sage{16-2}}  & 16 \\
 \hline
 \answer{\sage{22-2}}  & 22 \\
 \hline
  \answer{\sage{26-2}} & 26 \\
 \hline
  \answer{\sage{31-2}} & 31 \\
 \hline
 \mathbf{Rule:} & \mathbf{+\answer{2}} \\
 \hline
\end{array}
$
\hfill
$
\begin{array}{|r|r|}
\hline
 IN & OUT \\
 \hline
 5 & 10 \\
 \hline
  \answer{\sage{14-5}} & 14 \\
 \hline
 \answer{\sage{19-5}}  & 19 \\
 \hline
 \answer{\sage{22-5}}  & 22 \\
 \hline
 \answer{\sage{26-5}}  & 26 \\
 \hline
 \answer{\sage{31-5}}  & 31 \\
 \hline
 \textbf{Rule:} & \mathbf{+\answer{5}} \\
 \hline
\end{array}
$
\hfill
$
\begin{array}{|r|r|}
\hline
 IN & OUT \\
 \hline
 75 & 85 \\
 \hline
 \answer{\sage{113-10}}  & 113 \\
 \hline
 \answer{\sage{235-10}}  & 235 \\
 \hline
 \answer{\sage{650-10}}  & 650 \\
 \hline
 \answer{\sage{742-10}}  & 742 \\
 \hline
 \answer{\sage{900-10}}  & 900 \\
 \hline
 \textbf{Rule:} & \mathbf{+\answer{10}} \\
 \hline
\end{array}
$
\end{problem}

\begin{problem}
How many rings does everyone have?
\\
\\
% first column
\begin{minipage}[c]{0.5\textwidth}
\begin{itemize}
    \item Ms. Laurie has 12 rings. 
    \item Ms. Brandy has 3 more than Ms. Laurie.
    \item Ms. Sarah has 4 less than Ms. Brandy.
    \item Ms. Sonia has double the amount of Ms. Laurie.
    \item Ms. Marjon has 5 less than Ms. Sonia.
    \item Ms. Maryam has 2 more than Ms. Sarah.
\end{itemize}
\end{minipage}
\hfill
%second column
\begin{minipage}[c]{0.3\textwidth}
%\setlength\extrarowheight{12pt}
$
\begin{array}{|l|c|}
\hline 
Ms.~ Laurie &  \answer{\sage{12}} \\
\hline 
Ms.~ Brandy &  \answer{\sage{12+3}}\\
\hline
Ms.~ Sarah &  \answer{\sage{12+3-4}}\\
\hline
Ms.~ Sonia &  \answer{\sage{2*12}}\\
\hline
Ms.~ Marjon &  \answer{\sage{2*12-5}}\\
\hline
Ms.~ Maryam &  \answer{\sage{12+3-4+2}}\\
\hline
\end{array}
$
\end{minipage}
\end{problem}
\end{document}
%\OnehalfSpacing

\section*{\color{olive}Counting I}

Please, count the total number of the black dots below. Make sure that you count by groups, not one-by-one!

\begin{enumerate}%[$(a)$]
    \item \drawdie{1}+\drawdie{3}+\drawdie{3}+\drawdie{3}+\drawdie{3}+\drawdie{3}+\drawdie{3}+\drawdie{3}+\drawdie{3}+\drawdie{3}+\drawdie{3}
    Total= $\answer{31}$
    \item \drawdie{2}+\drawdie{3}+\drawdie{3}+\drawdie{3}+\drawdie{3}+\drawdie{3}+\drawdie{3}+\drawdie{3}+\drawdie{3}+\drawdie{3}+\drawdie{3}
    Total= $\begin{bmatrix} ~ & ~ ~~~~\end{bmatrix}$
    
    \item \drawdie{4}+\drawdie{4}+\drawdie{4}+\drawdie{4}+\drawdie{4}+\drawdie{4}+\drawdie{4}+\drawdie{4}+\drawdie{4}+\drawdie{4}+\drawdie{4}+\drawdie{4}+\drawdie{4}
    Total=$\begin{bmatrix} ~ &~~~ ~~~~ \end{bmatrix}$

    \item \drawdie{6}+\drawdie{6}+\drawdie{6}+\drawdie{6}+\drawdie{6}+\drawdie{6}+\drawdie{6}+\drawdie{6}+\drawdie{6}+\drawdie{6}+\drawdie{6}+\drawdie{6}+\drawdie{6}
    Total=$\begin{bmatrix} ~ &~~~ ~~~~ \end{bmatrix}$

   \item \drawdie{7}+\drawdie{7}+\drawdie{7}+\drawdie{7}+\drawdie{7}+\drawdie{7}+\drawdie{7}+\drawdie{7}+\drawdie{7}+\drawdie{7}+\drawdie{7}+\drawdie{7}
    Total=$\begin{bmatrix} ~ &~~~ ~~~~ \end{bmatrix}$   
  \item \drawdie{8}+\drawdie{3}+\drawdie{3}+\drawdie{7}+\drawdie{9}+\drawdie{3}+\drawdie{7}+\drawdie{5}+\drawdie{5}+\drawdie{2}+\drawdie{8}+\drawdie{3}+\drawdie{3}+\drawdie{7}+\drawdie{4}+\drawdie{6}+\drawdie{2}+\drawdie{8}+\drawdie{3}+\drawdie{3}+\drawdie{7}+\drawdie{3}+\drawdie{8}+\drawdie{3}+\drawdie{4}+\drawdie{5}+\drawdie{6}+\drawdie{3}+\drawdie{5}+\drawdie{3}+\drawdie{2}+\drawdie{8}+\drawdie{3}+\drawdie{3}+\drawdie{7}+\drawdie{4}+\drawdie{2}
    \\
    Total=$\begin{bmatrix} ~ &~~~ ~~~~ \end{bmatrix}$
\end{enumerate}


\pagebreak

\section*{\color{olive}Addition I}
If you draw a picture, use ``{\Large $\mathbf{\cdot=1,~|=10,~\square=100}$, \mancube $=1,000$, $\star=10,000$}''
    
    \begin{tabular}{|c|c|}
        \hline
        & ~~~~~~+6 ~~~~~\\
        \hline
        8 & \\
        \hline
        99 & \\
        \hline
        184 & \\
        \hline
        4007 & \\
        \hline
    \end{tabular}
    \hfill 
    \begin{tabular}{|c|c|}
        \hline
        & ~~~~~~+60 ~~~~~\\
        \hline
        8 & \\
        \hline
        99 & \\
        \hline
        184 & \\
        \hline
        4007 & $\answer{4067}$ \\
        \hline
    \end{tabular}
    \hfill \,
    \vfill
    \begin{tabular}{|c|c|}
        \hline
        & ~~~~~~+600 ~~~~~\\
        \hline
        8 & \\
        \hline
        99 & \\
        \hline
        184 & \\
        \hline
        4007 & \\
        \hline
    \end{tabular}    
    \hfill
    \begin{tabular}{|c|c|}
        \hline
        & ~~~~~~+6000 ~~~~~\\
        \hline
        8 & \\
        \hline
        99 & \\
        \hline
        184 & \\
        \hline
        4007 & \\
        \hline
    \end{tabular}
    \hfill \,
    \vfill
\end{document}
\newpage

\section*{\color{olive}Counting II}

Please, count the total number of the black dots below. Make sure that you count by groups, not one-by-one!

\begin{enumerate}[$(a)$]
    \item \drawdie{3}+\drawdie{5}+\drawdie{4}+\drawdie{2}+\drawdie{3}+\drawdie{7}+\drawdie{6}+\drawdie{3}+\drawdie{2}+\drawdie{9}+\drawdie{4}+\drawdie{2}\\
    Total= $\begin{bmatrix} ~ & ~ ~~~~\end{bmatrix}$
    
    \item \drawdie{1}+\drawdie{9}+\drawdie{3}+\drawdie{7}+\drawdie{6}+\drawdie{5}+\drawdie{4}+\drawdie{3}+\drawdie{2}+\drawdie{7}+\drawdie{2}+\drawdie{4}+\drawdie{8}+\drawdie{2}+\drawdie{5}+\drawdie{5}+\drawdie{4}+\drawdie{2}+\drawdie{8}+\drawdie{3}+\drawdie{6}+\drawdie{2}+\drawdie{4}+\drawdie{2}
    \\
    Total=$\begin{bmatrix} ~ &~~~ ~~~~ \end{bmatrix}$

    \item \drawdie{1}+\drawdie{2}+\drawdie{9}+\drawdie{7}+\drawdie{5}+\drawdie{4}+\drawdie{4}+\drawdie{6}+\drawdie{2}+\drawdie{4}+\drawdie{3}+\drawdie{4}+\drawdie{7}+\drawdie{3}+\drawdie{4}+\drawdie{3}+\drawdie{4}+\drawdie{5}+\drawdie{6}+\drawdie{3}+\drawdie{5}+\drawdie{3}+\drawdie{4}+\drawdie{4}
    \\
    Total=$\begin{bmatrix} ~ &~~~ ~~~~ \end{bmatrix}$

   \item \drawdie{1}+\drawdie{2}+\drawdie{4}+\drawdie{3}+\drawdie{2}+\drawdie{3}+\drawdie{9}+\drawdie{2}+\drawdie{3}+\drawdie{4}+\drawdie{3}+\drawdie{2}+\drawdie{3}+\drawdie{3}+\drawdie{3}+\drawdie{3}+\drawdie{5}+\drawdie{3}+\drawdie{2}+\drawdie{3}+\drawdie{3}+\drawdie{4}+\drawdie{3}+\drawdie{3}+\drawdie{3}+\drawdie{3}+\drawdie{3}+\drawdie{5}+\drawdie{5}+\drawdie{3}+\drawdie{2}+\drawdie{3}+\drawdie{2}+\drawdie{3}+\drawdie{3}+\drawdie{6}+\drawdie{3}+\drawdie{1}+\drawdie{3}
    \\
    Total=$\begin{bmatrix} ~ &~~~ ~~~~ \end{bmatrix}$
\end{enumerate}
\newpage

\section*{\color{olive}Addition Tables II}

If you draw a picture, use ``{\Large $\mathbf{\cdot=1,~|=10,~\square=100}$, \mancube $=1,000$, $\star=10,000$}''  

    \begin{tabular}{|c|c|}
        \hline
        & ~~~~~~+8 ~~~~~\\
        \hline
        7 & \\
        \hline
        73 & \\
        \hline
        68 & \\
        \hline
        97 & \\
        \hline
        118 & \\
        \hline
        182 & \\
        \hline
        3884 & \\
        \hline
        8765 & \\
        \hline
    \end{tabular}
    \hfill
    \begin{tabular}{|c|c|}
        \hline
        & ~~~~~~+80 ~~~~~\\
        \hline
        7 & \\
        \hline
        73 & \\
        \hline
        68 & \\
        \hline
        97 & \\
        \hline
        118 & \\
        \hline
        182 & \\
        \hline
        3884 & \\
        \hline
        8765 & \\
        \hline
    \end{tabular}
    
    \vfill
    
    \begin{tabular}{|c|c|}
        \hline
        & ~~~~~~+800 ~~~~~\\
        \hline
        7 & \\
        \hline
        73 & \\
        \hline
        68 & \\
        \hline
        97 & \\
        \hline
        118 & \\
        \hline
        182 & \\
        \hline
        3884 & \\
        \hline
        8765 & \\
        \hline
    \end{tabular}
    \hfill
    \begin{tabular}{|c|c|}
        \hline
        & ~~~~~~+8000 ~~~~~\\
        \hline
        7 & \\
        \hline
        73 & \\
        \hline
        68 & \\
        \hline
        97 & \\
        \hline
        118 & \\
        \hline
        182 & \\
        \hline
        3884 & \\
        \hline
        8765 & \\
        \hline
    \end{tabular}
    
    \newpage
    
    
    
    
    
\section*{\color{olive} Numbers Between I}
 
\begin{enumerate}[$(a)$]
    \item Write a number greater than 50. $\begin{bmatrix} ~ &~~~ ~~~~ \end{bmatrix}$
    \vfill    \item Write a number less than 100. $\begin{bmatrix} ~ &~~~ ~~~~ \end{bmatrix}$
    \vfill
     \item Write a number greater than 50 and less than 100. $\begin{bmatrix} ~ &~~~ ~~~~ \end{bmatrix}$
    \vfill
    \item Write a number greater than 50 and less than 100. This time, it must have 5 on its TENs-digit. $\begin{bmatrix} ~ &~~~ ~~~~ \end{bmatrix}$
    \vfill
    \item Write a number greater than 50 and less than 100. This time, it must have 5 on its ONEs-digit. $\begin{bmatrix} ~ &~~~ ~~~~ \end{bmatrix}$
    \vfill
\end{enumerate}
\newpage 
\section*{\color{olive}Counting III}
Please, count the number of black dots. Make sure that you count by groups, not one-by-one!

\begin{enumerate}[$(a)$]
    \item \drawdie{9}+\drawdie{5}+\drawdie{5}+\drawdie{7}+\drawdie{6}+\drawdie{2}+\drawdie{4}+\drawdie{5}+\drawdie{3}+\drawdie{9}+\drawdie{3}+\drawdie{2}\\
    Total= $\begin{bmatrix} ~ & ~ ~~~~\end{bmatrix}$
    
    \item \drawdie{1}+\drawdie{2}+\drawdie{3}+\drawdie{4}+\drawdie{5}+\drawdie{6}+\drawdie{7}+\drawdie{1}+\drawdie{2}+\drawdie{3}+\drawdie{4}+\drawdie{5}+\drawdie{6}+\drawdie{7}+\drawdie{1}+\drawdie{2}+\drawdie{3}+\drawdie{4}+\drawdie{5}+\drawdie{6}+\drawdie{7}+\drawdie{1}+\drawdie{2}+\drawdie{3}
    \\
    Total=$\begin{bmatrix} ~ &~~~ ~~~~ \end{bmatrix}$

    \item \drawdie{1}+\drawdie{5}+\drawdie{9}+\drawdie{4}+\drawdie{7}+\drawdie{1}+\drawdie{4}+\drawdie{2}+\drawdie{6}+\drawdie{1}+\drawdie{5}+\drawdie{8}+\drawdie{2}+\drawdie{5}+\drawdie{3}+\drawdie{7}+\drawdie{2}+\drawdie{6}+\drawdie{9}+\drawdie{3}+\drawdie{6}+\drawdie{4}+\drawdie{8}+\drawdie{3}
    \\
    Total=$\begin{bmatrix} ~ &~~~ ~~~~ \end{bmatrix}$

   \item \drawdie{1}+\drawdie{5}+\drawdie{3}+\drawdie{4}+\drawdie{3}+\drawdie{2}+\drawdie{9}+\drawdie{2}+\drawdie{5}+\drawdie{3}+\drawdie{1}+\drawdie{3}+\drawdie{3}+\drawdie{3}+\drawdie{3}+\drawdie{5}+\drawdie{3}+\drawdie{2}+\drawdie{3}+\drawdie{2}+\drawdie{3}+\drawdie{4}+\drawdie{5}+\drawdie{3}+\drawdie{3}+\drawdie{3}+\drawdie{3}+\drawdie{5}+\drawdie{5}+\drawdie{5}+\drawdie{3}+\drawdie{5}+\drawdie{3}+\drawdie{2}+\drawdie{3}+\drawdie{6}+\drawdie{5}+\drawdie{3}+\drawdie{2}
    \\
    Total=$\begin{bmatrix} ~ &~~~ ~~~~ \end{bmatrix}$
\end{enumerate}

\newpage
\section*{\color{olive}Numbers between II}
{\parskip=0cm
    \begin{enumerate}[$(a)$]
    \item Write a number greater than 45. 
    $\begin{bmatrix} ~ &~~~ ~~~~ \end{bmatrix}$
    \vfill    
    \item Write a number less than 55.
    $\begin{bmatrix} ~ &~~~ ~~~~ \end{bmatrix}$
    \vfill    
    \item Write a number greater than 45 and less than 55.
    $\begin{bmatrix} ~ &~~~ ~~~~ \end{bmatrix}$
    \vfill    
    \item Write an \textbf{odd} number greater than 45 and less than 55.
    $\begin{bmatrix} ~ &~~~ ~~~~ \end{bmatrix}$
    \vfill    
    \item Write a number greater than 45 and less than 55. This time, it should have 4 in the 10s place.
    $\begin{bmatrix} ~ &~~~ ~~~~ \end{bmatrix}$
    \vfill    
    \item Write a number greater than 45 and less than 55. This time, it should have 4 in the 1s place.
    $\begin{bmatrix} ~ &~~~ ~~~~ \end{bmatrix}$
    \vfill    
\end{enumerate}
}

\newpage
\section*{\color{olive}Mental Addition I}

{\parskip=0cm
Please, use ``{\Large $\mathbf{\cdot}$}'' for 1s, ``$\mathbf{|}$'' for 10s, ``{\Large$\mathbf{\square}$}'' for 100s and ``{\Large \mancube}'' for 1,000s to solve the following problems. Solve the problem in the following order:
\begin{enumerate}[$(S1)$]
  \setlength\itemsep{0em}
    \item First, fill in the circles. 
    \item Secondly, fill in the ellipse so that each picture occurs at most 9 times.
    \item Finally, use the information in the ellipse to write the sum into the rectangle as a number. 
\end{enumerate}
    \noindent\makebox[\linewidth]{\rule{\textwidth}{1pt}} 
    \mentaladd{3}{56}
    \\
    \noindent\makebox[\linewidth]{\rule{\textwidth}{1pt}} 
    \mentaladd{30}{56}
    \\
    \noindent\makebox[\linewidth]{\rule{\textwidth}{1pt}} 
    \mentaladd{35}{52}
\newpage
    \mentaladd{44}{53}
    \\
    \noindent\makebox[\linewidth]{\rule{\textwidth}{1pt}} 
    \mentaladd{15}{54}
    \\
    \noindent\makebox[\linewidth]{\rule{\textwidth}{1pt}} 
    \mentaladd{76}{40}
    \\
    \noindent\makebox[\linewidth]{\rule{\textwidth}{1pt}} 
    \mentaladd{38}{55}
\newpage
    \mentaladd{27}{39}
    \\
    \noindent\makebox[\linewidth]{\rule{\textwidth}{1pt}} 
    \mentaladd{57}{44}
    \\
    \noindent\makebox[\linewidth]{\rule{\textwidth}{1pt}} 
    \mentaladd{89}{48}
    \\
    \noindent\makebox[\linewidth]{\rule{\textwidth}{1pt}} 
    \mentaladd{77}{44}
\newpage    
    \mentaladd{888}{4}
    \\
    \noindent\makebox[\linewidth]{\rule{\textwidth}{1pt}} 
    \mentaladd{888}{40}
    \\
    \noindent\makebox[\linewidth]{\rule{\textwidth}{1pt}} 
    \mentaladd{888}{400}
    \\
    \noindent\makebox[\linewidth]{\rule{\textwidth}{1pt}} 
    \mentaladd{888}{44}
}
\end{document}
