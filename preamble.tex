%% You can put user macros here
%% However, you cannot make new environments

\usepackage[letterpaper, total={6in, 8in}]{geometry}
\graphicspath{{./}{firstExample/}{secondExample/}}

\usepackage{tikz}
\usepackage{tkz-euclide}
\usetkzobj{all}

\tikzstyle geometryDiagrams=[ultra thick,color=blue!50!black]

%\usepackage{enumerate}
\usepackage{euler}
\usepackage{manfnt}
\usepackage{MnSymbol}%

\usepackage{geometry}
 \geometry{
 letterpaper,
 left=18mm,
 right=18mm,
 top=20mm,
 bottom=14mm
 }
 
 
\usetikzlibrary{shapes,snakes}
\tikzset{
  dot hidden/.style={},
  line hidden/.style={},
  dot colour/.style={dot hidden/.append style={color=#1}},
  dot colour/.default=black,
  line colour/.style={line hidden/.append style={color=#1}},
  line colour/.default=black
}

%%%%%%%%%%%%%%% DICE! %%%%%%%%%%%%%%%%%
\def\dotsize{0.1}

\usepackage{xparse}
\tikzset{
  dot hidden/.style={},
  line hidden/.style={},
  dot colour/.style={dot hidden/.append style={color=#1}},
  dot colour/.default=black,
  line colour/.style={line hidden/.append style={color=#1}},
  line colour/.default=black
}

\usepackage{xparse}
\NewDocumentCommand{\drawdie}{O{}m}{
\begin{tikzpicture}[x=2em,y=2em,radius=0.1,#1]
  \draw[rounded corners=0.5,line hidden] (0,0) rectangle (1,1);
  \ifodd#2
    \fill[dot hidden] (0.5,0.5) circle;
  \fi
  \ifnum#2>1
    \fill[dot hidden] (0.2,0.2) circle;
    \fill[dot hidden] (0.8,0.8) circle;
   \ifnum#2>3
     \fill[dot hidden] (0.2,0.8) circle;
     \fill[dot hidden] (0.8,0.2) circle;
    \ifnum#2>5
      \fill[dot hidden] (0.8,0.5) circle;
      \fill[dot hidden] (0.2,0.5) circle;
     \ifnum#2>7
       \fill[dot hidden] (0.5,0.8) circle;
       \fill[dot hidden] (0.5,0.2) circle;
      \fi
    \fi
  \fi
\fi
\end{tikzpicture}
}   

%%%%%%%%%%%%%%%%%

%%%%%%%%%%%%%

\newcommand{\Hsquare}{
\boxed{\begin{matrix} ~~~~~~~ \Huge\, \end{matrix}}
}

%%%%%%%%%%%%%%%%%%%%%% MENTAL ADD BEGIN%%%
\tikzset{
mycircle/.pic={
\draw[dashed] (0,0) circle (1.6cm); 
}
}

\tikzset{
myellipse/.pic={
\draw[dashed] (0,0) ellipse (3.2cm and 1.6cm); 
}
}

\tikzset{
mybox/.pic={
\draw[line width=0.35mm] (-1.6,-.8) rectangle (1.6,.8);  
}  
}

\newcommand{\mentaladd}[2]{
\begin{tikzpicture}
  \matrix[column sep=3mm, row sep=3mm,ampersand replacement=\&]{
      \node[]{\Large$\mathbf{#1}$}; \& \node[]{\Large $\mathbf{+}$}; \& \node[]{\Large $\mathbf{#2}$}; \&\node[]{\Large $\mathbf{=}$}; \& \pic{mybox} ; \\
      \pic{mycircle}; \& \node[]{\Large +};\& \pic{mycircle}; \&\node[]{\Large=}; \& \pic{myellipse}; \\
    };
\end{tikzpicture}
}
%%%%%%%%%%%%%%%%%%MENTAL ADD END %%%%%%%%%%%%%

%%%%%%%%%%%%%%%%%%%% KINDERGARTEN WRITING LINE 

\usepackage{scalerel}
\usepackage[usestackEOL]{stackengine}
\newlength\letterheight
\setlength\letterheight{.5in}
\def\midpitch{.6}
\parskip 0.3in
\def\stacktype{L}\def\stackalignment{l}
%From morsburg at http://tex.stackexchange.com/questions/12537/
%how-can-i-make-a-horizontal-dashed-line/12553#12553
\def\dashfill{\cleaders\hbox to .6em{-}\hfill}
\newcommand\dashline[1]{\abovebaseline[-2pt]{\hbox to #1{\dashfill\hfil}}}
\def\myhline{\rule{\textwidth}{.3pt}}
\newcommand\blankrow{%
  \setstackgap{L}{\the\letterheight}%
  \stackon[\midpitch\letterheight]{%
    \stackon{\myhline}{\myhline}}{\dashline{\textwidth}}%
}
\newcommand\fillrow[1]{%
  \stackinset{l}{}{b}{}%
  {\scalerel*{\rule{0pt}{.85\ht\strutbox}\smash{#1}}{\blankrow}}{\blankrow}}
%%%%%%%%%%%%%%%%%%%%%%%% END KINDERGARTEN
